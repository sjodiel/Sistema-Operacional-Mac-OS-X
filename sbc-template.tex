\documentclass[12pt]{article}

\usepackage{sbc-template}
\usepackage{natbib}
\bibliographystyle{unsrtnat}
\usepackage{graphicx,url}

\usepackage[brazil]{babel}   
%\usepackage[latin1]{inputenc}  
\usepackage[utf8]{inputenc}  
% UTF-8 encoding is recommended by ShareLaTex

     
\sloppy

\title{Sistema Operacional MAC OS X}

\author{Jodiel Fabricio Dias dos Santos\inst{1} }


\address{Departamento de Informática -- Universidade Federal do Maranhão
  (UFMA)\\
  Cidade Universitária -- São Luís -- MA -- Brasil
  \email{jodielfabricio@gmail.com}
}

\begin{document} 

\maketitle

\begin{abstract}
  Summary of key features of the MAC OS X operating system.
\end{abstract}

\begin{resumo} 
  Resumo das principais características do Sistema Operacional MAC OS X.
\end{resumo}

\section{Gerenciamento de Processos}\label{sec:firstpage}
 
\subsection{Escalonamento de Processos}
O escalonador de processos do MAC OS X é derivado do escalonador OSFMK, porém várias modificações foram feitas para o tratamento de interatividade no projeto de escalonador base.

Assim como o kernel do FreeBSD, o escalonador de processos do Mac OS X escalona os processos baseado numa variante do algoritmo de múltiplas filas com realimentação, porém divide as filas de prioridades em 4 grupos de processos, agrupados de acordo com suas características, conforme é descrito na Tabela~\ref{tabela1}.
\begin{table}[ht]
\begin{tabular}{|c|c|}
\hline
\textbf{Grupo de Filas} & \textbf{Características} \\ \hline
Normal & Prioridade das aplicações normais do usuário \\ \hline
Sistema & \begin{tabular}[c]{@{}c@{}}Prioridade das aplicações do sistema, com prioridade superior \\ as threads normais de usuário\end{tabular} \\ \hline
Kernel & \begin{tabular}[c]{@{}c@{}}Classe reservada para threads em espaço de kernel que necessitam rodar\\ em uma prioridade superior as threads de sistema (como a thread de\\ descalonamento de E/S, por exemplo)\end{tabular} \\ \hline
Tempo Real & \begin{tabular}[c]{@{}c@{}}Threads nas quais a prioridade está baseada na necessidade de reservar\\ uma fração pré-definida de ciclos de clock, independente de outras\\ atividades sendo executadas no sistema\end{tabular} \\ \hline
\end{tabular}
\caption{Grupos de Filas de Processos no Escalonador do MAC OS X}
\label{tabela1}
\end{table}

No caso de tarefas de tempo real, é feito um controle forte pela aplicação no número de ciclos do processador a serem utilizados. Assim, o programador utiliza chamadas de sistema para requisitar ao sistema rodar por A ciclos nos próximos B ciclos. 

\subsection{Mach Scheduling}
\begin{itemize}
    \item Suporta Tempo-Compartilhado e Prioridade-Fixa;
    \item Prioridades: Normal, Alta-Prioridade de Sistema, Somente modo-Kernel e Theads de Tempo-Real;
    \item Theads de prioridade de Tempo-Real geralmente são de prioridade Fixa;
    \item Threads de prioridade fixa executam por um pré-determinado período de tempo ou até que uma thread de prioridade maior queira executar. Então ela é colocada no fim da fila daquela prioridade (normal);
    \item Threads de Tempo-Real: Uma thread pode dizer ao agendador que ela precisa rodar por 3000 ciclos dos próximos 7000, além de dizer se precisam ser contínuos ou não. Obviamente, longo ciclos contínuos podem ser custosos, mas também necessários;
    \item Se uma thread de tempo-real não respeita o tempo de processamento indicada ela é penalizada e pode até ser rebaixada a uma prioridade “normal”;
    \item Threads que usam muito tempo do processador recebem uma prioridade baixa para evitar que threads de prioridade alta monopolizem o processador. As de prioridade alta devem executar rapidamente e liberar o recurso.
\end{itemize}
\subsection{Comunicação Inter-Processos}
\begin{itemize}
    \item Tarefas clientes acessam serviços de tarefas servidoras por um canal de comunicação (ports);
    \item Tarefas mantém, repassam e obtém direitos sobre os ports;
    \item As “pontas” destes canais são chamadas ports. Os port-rights denotam as permissões para usar estes canais. Eles são o mecanismo fundamental de segurança no Mach. Ter um right, é ter a capacidade de acessar ou manipular um objeto.
    \item Comunicação por: filas de mensagens, semáforos, notificações, lock sets e RPCs (Remote Procedure Calls).
    \item Várias tarefas podem manter o direito de enviar (write) mensagens para filas, mas só uma pode ter o direito de lê-la. Mensagens podem ser dados puros, cópias de memory ranges, direitos de ports, e mais... (são assíncronas).
    \item Semáforos suportam wait, post e post all.
    \item Lock Sets: mutex sobre uma seção crítica.
\end{itemize}

\section{Gerenciamento de Memória}
\subsection{Características Principais}
\begin{itemize}
    \item Paginação sob demanda;
    \item External Memory Management Interface (EMMI);
    \item Named Memory Entries;
    \item Lazy Evaluation de Memória Copiada (Shadows Objects);
    \item Memory Maps;
    \item Herança de Named Regions;
    \item UPLs.
\end{itemize}

No MAC OS X, cada processo tem o seu próprio conjunto (32bits ou 64 bits) de espaço de endereço virtual. Para processos de 32 bits, cada processo tem um espaço que pode endereçar dinamicamente chegando ao limite de 4GB. Para processo de 64 bits, cada processo pode endereçar dinamicamente até o limite de 18 exabytes.

Além do VM Subsystem baseado no Core-Mach, o gerenciamento de memória do MAC OSX engloba diversos outros mecanismos, alguns dos quais não são partes estritamente do sistema VM, mas estão intimamente relacionados.

O Subsistema VM do kernel Mach consiste do módulo machine-dependent phisical map (pmap) e outros módulos machine-independent para gerenciar a estrutura de dados, tais como os virtual address space maps (VM maps). O Kernel exporta diversas rotinas para o Espaço do Usuário (user space) como parte de Mach VM API.

O kernel usa o UPL (Universal Page List), Estrutura de Dados para descrever como delimitar um conjunto de páginas físicas. O UPL é criado baseado na associação das páginas (pages) com o Objetos VM (objects VM). UPLs incluem vários atributos das páginas que descrevem. Subsistemas do Kernel, particularmente o File System, usam UPLs para se comunicar com o subsistema VM.

O UBC (Unified Buffer Cache) constitui um conjunto de páginas para armazenar (caching) o conteúdo dos arquivos e as porções anônimas do espaço de endereços. Memória anônima não é sustentada por arquivos regulares, dispositivos, ou mesmo alguma outra fonte de memória e o exemplo mais comum é a memória alocada dinamicamente. 

O Kernel inclui 3 paginadores internos: o paginador padrão (anônimo), o paginador para dispositivos e o paginador para vnode. Eles tratam operações de entrada e saída sobre regiões da memória. Os paginadores comunicam-se com o Subsistema Mach-VM utilizando interfaces UPL e derivadas do paginador do Mach.

Ao contrário da maioria dos sistemas operacionais baseados em UNIX, o MAC OS X não usa uma pré-partição swap para a memória virtual. Em vez disso, ele usa todo o espaço disponível na máquina da partição de
boot.
\subsection{Memória Compartilhada}
Memória compartilhada é a memória que pode ser escrita ou lida a partir de dois ou mais processos. Memória partilhada pode ser herdada a partir de um processo-mãe, criado por um servidor de memória compartilhada, ou explicitamente criado por um pedido de exportação para outras aplicações. O uso da memória compartilhada/partilhada incluem no MAC OS X:

\begin{itemize}
    \item Compartilhando grandes recursos, como ícones ou sons;
    \item Rápida a comunicação entre um ou mais processos.
\end{itemize}
\subsection{Wired Memory}
Wired memory (também chamado de memória residente) armazena código kernel e estruturas de dados, que nunca deve ser paginada para o disco.
Cada thread criada, cada subprocesso bifurcado, e cada biblioteca ligada
contribui para a ocupação da memória residente do sistema.
As medidas mudam a cada nova versão do MAC OS X.

Além das solicitações de Wired-Memory geradas no nível do usuário, as seguintes entidades do kernel utilizam wired-memory:
\begin{itemize}
    \item VM objetos;
    \item A memória virtual buffer cache;
    \item I/O buffer caches;
    \item Condutores
\end{itemize}
\section{Gerenciamento de E/S}

O Mac OS X fornece um ambiente para interface com dispositivos de entrada e saída, além de uma arquitetura para o desenvolvimento de drivers. O conjunto ambiente e arquitetura é chamado de \textit{I/O Kit}. 

A arquitetura da I/O Kit fornece uma interface orientada a objetos em um subconjunto de C++ (Embedded C++) para modelar a hierarquia de hardware em software. A funcionalidade comum a diferentes dispositivos é encapsulada em classes (também chamadas de famílias), que são herdadas pelos drivers desses dispositivos.

A I/O Kit também fornece um ambiente de execução protegido para drivers. Esse ambiente é baseado na ideia de que drivers podem ser dinamicamente carregados e descarregados. Assim, todo driver deve gerenciar seu ciclo de vida. Ainda, quando um dispositivo é encontrado, o sistema deve procurar um driver que corresponda àquele dispositivo. Essa função é também realizada pelos nubs. 

O I/O Kit fornece ainda uma interface para controle de dispositivos em nível de usuário. Isso permite que o uso de memória do kernel seja minimizado, o que é importante porque o kernel deve sempre estar na memória física. Essencialmente, o usuário especifica em seu programa que dispositivo ele deseja acessar. O kernel então procura o driver correspondente (usando os nubs) e executa a função requisitada de uma maneira controlada. 


\section{Sistema de Arquivos}
\subsection{HFS+}
O sistema de arquivos padrão do MAC OS X é o HFS+ (Hierarchical File System Plus, ou Sistema de Arquivos Hierárquico Mais). Sua instância fundamental é o volume, que pode estar em um disco, uma parte deste ou vários discos em conjunto. Os blocos de alocação são a unidade elementar onde são alocados os arquivos, e seu tamanho corresponde a um múltiplo do tamanho do setor do dispositivo utilizado, sendo o mais comum 512 bytes, o que contrasta com o bloco de alocação padrão do MAC OS X, que é de 4KB, sendo que este pode chegar a 1MB (tamanho definido artificialmente no arquivo /sys/param.h). O espaço de armazenamento usa números de bloco de alocação de 32 bits com alocação postergada de blocos físicos.

Quanto a fragmentação, a implementação do UFS herdada do sistema BSD presente no HFS+ emprega uma unidade adicional, complementando o bloco, chamado fragmento, que é uma fração deste a ser compartilhada entre arquivos.

O HFS+ é um sistema de arquivos que suporta nomes de arquivos e diretórios até 255 caracteres codificados em Unicode UTF-8 , arquivos de até 8 exabytes e múltiplos mecanismos de referenciação através de aliases, hard links e links simbólicos.

\subsection{Controle de Acesso e Permissões}
A partir da versão 10.4 do sistema operacional, são suportados pelo sistema de arquivos simultaneamente ACLs (listas de controle de acesso), que são armazenadas como atributos estendidos, permissões de arquivos no estilo Unix-Like e Flags de arquivo no estilo BSD (append-only, immutable, undeletable, etc.).

Todas as chamadas do gerenciador de arquivos do MAC OS X, no entanto, são subsidiadas pelo subsistema BSD, que como qualquer Unix-Like no que diz respeito a proteção de escrita, não provê nenhum lock para arquivos abertos por múltiplas aplicações. Existe aprovisionamento ainda para o armazenamento de múltiplos timestamps para cada objeto do sistema de arquivos.

\subsection{Bundles e Fork de Arquivo}
O sistema operacional suporta um tipo especial, chamado File Fork (Ramificação de Arquivo, em tradução livre) que encapsula vários arquivos dentro
de um único, tendo dois sempre presentes, o data fork e o resource fork, armazenados em árvores-B separadas. Outro conceito particular do MAC OS X são os Bundles. Estes são uma coleção de recursos relacionados empacotados como numa hierarquia de diretórios. 
\subsection{Outras Características }
\begin{itemize}
    \item Otimizações: Desfragmentação On-the-Fly e Zona de meta-dados;
    \item Camada VFS;
    \item O sistema de arquivos HFS+ não suporta sparse files (arquivos dispersos), que é a técnica de preencher com zeros espaços em branco que ficam entre arquivos de sistema, para que sejam vistos como um único arquivo, sem com isso ocupar mais espaço em disco.
\end{itemize}

%\bibliographystyle{sbc}
%\bibliography{sbc-template}
\bibliography{sbc2}
\nocite{apple}
\nocite{iokit}
\nocite{file}
\nocite{wiki}

\end{document}
